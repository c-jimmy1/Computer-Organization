\documentclass[12pt]{article}

\usepackage[shortlabels]{enumitem}
\usepackage[margin=1in,headheight=15pt]{geometry}   % Adjusted headheight
\usepackage{amsmath}
\usepackage{fancyhdr}
\usepackage{graphicx}
\usepackage{cancel}

% Set up fancy header/footer
\pagestyle{fancy}
\fancyhead[LO,L]{Jimmy Chen}
\fancyhead[CO,C]{CSCI 2500 - Computer Organization}
\fancyhead[RO,R]{October 2, 2023}
\fancyfoot[LO,L]{}
\fancyfoot[CO,C]{\thepage}
\fancyfoot[RO,R]{}
\renewcommand{\headrulewidth}{0.4pt}
\renewcommand{\footrulewidth}{0.4pt}

\begin{document}
\section{Homework 1}
\subsection{Textbook Problem 4.15.5}
Consider three different processors P1, P2, and P3 executing the same instruction set. 
P1 has a 3 GHz clock rate and a CPI of 1.5. P2 has a 2.5 GHz clock rate and a CPI of 1.0. 
P3 has a 4.0 GHz clock rate and has a CPI of 2.2.

\begin{center}
    \begin{tabular}{|c|c|c|c|}
        \hline
        Processor & Clock Rate & CPI & Performance \\
        \hline
        P1 & 3 GHz & 1.5 & 2 GHz \\
        \hline
        P2 & 2.5 GHz & 1.0 & 2.5 GHz \\
        \hline
        P3 & 4.0 GHz & 2.2 & 1.8 GHz \\
        \hline
    \end{tabular}
\end{center}

\begin{enumerate}[(a)]
    \item Which processor has the highest performance expressed in instructions per second?
    \textbf{Answer:}
    \\
    Instructions per Second = Clock Rate/CPI
        \begin{center}
            P1 = 3 GHz/1.5 = 2 GHz\\
            P2 = 2.5 GHz/1.0 = 2.5 GHz\\
            P3 = 4.0 GHz/2.2 = 1.8 GHz\\
        \end{center}
    \textbf{P2 has the highest performance expressed in instructions per second}
    
    \item If the processors each execute a program in 10 seconds, find the number of cycles and the number of instructions.
    \\
    \textbf{Answer:}
    \\
    Number of Cycles = Clock Rate * Execution Time
        \begin{center}
            P1 = 3 GHz * 10s = 30 Billion Cycles\\
            P2 = 2.5 GHz * 10s = 25 Billion Cycles\\
            P3 = 4.0 GHz * 10s = 40 Billion Cycles\\
        \end{center}
    
    Number of Instructions = Instructions per Second * Execution Time
        \begin{center}
            P1 = 2 Hz * 10s = 20 Billion Instructions\\
            P2 = 2.5 GHz * 10s = 25 Billion Instructions\\
            P3 = 1.8 GHz * 10s = 18 Billion Instructions\\
        \end{center}

    \item We are trying to reduce the execution time by 30\% but this leads to an increase of 20\% in the CPI. What clock rate should we have to get this time reduction?
    \\
    \textbf{Answer:}
    \\
    New CPI = 1.2 * Old CPI
    New Execution Time = 0.7 * Old Execution Time
    Clock Rate = (CPI * Instructions per Second) / Execution Time
        \begin{center}
            P1 = (20 * 1.8) / 7 $\approx$ 5.14 GHz\\
            P2 = (25 * 1.2) / 7 $\approx$ 4.29 GHz\\
            P3 = (18 * 2.64) / 7 $\approx$ 6.79 GHz\\
        \end{center}

\end{enumerate}

\subsection{Textbook Problem 4.15.8}
Compilers can have a profound impact on the performance of an application. Assume that for a program, compiler A results in a dynamic instruction count of 1.0E9 and has an execution time of 1.1 s, while compiler B results in a dynamic instruction count of 1.2E9 and an execution time of 1.5 s.

\begin{enumerate}[(a)]
    \item Find the average CPI for each program given that the processor has a clock cycle time of 1 ns.
    \item Assume the compiled programs run on two different processors. If the execution times on the two processors are the same, how much faster is the clock of the processor running compiler A's code versus the clock of the processor running compiler B's code?
    \item A new compiler is developed that uses only 6.0E8 instructions and has an average CPI of 1.1. What is the speedup of using this new compiler versus using compiler A or B on the original processor?


\end{enumerate}



\subsection{Textbook Problem 4.15.10}
\subsection{Textbook Problem 4.15.13}
\subsection{Textbook Problem 4.15.14}
\subsection{Textbook Problem 4.15.15}
\subsection{Textbook Problem 4.15.16}


\end{document}
